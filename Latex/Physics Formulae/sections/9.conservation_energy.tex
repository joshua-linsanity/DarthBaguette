\documentclass[../PhysicsFormulae.tex]{subfiles}
\begin{document}
\subsection{Work Done on a System}
We can extend the work-energy theorem to be more general, given by
\[ W = \Delta K + \Delta U + \Delta E_{int} \]
where $E_{int}$ is the sum of energies at the microscopic scale, from the kinetic energy of randomly moving particles to the potential energy stored in bonds.
 
\subsection{Frictional Work}
Frictional work is \textit{not} given by $W_f = fs$ as sliding friction cannot be treated as acting at a point from an energy viewpoint.\\
Imagine a book being pulled along a table at constant speed with a rope, given by
\[ W_T + W_f = \Delta E_{int} \]
which becomes
\[ W_f = -W_T + \Delta E_{int} = -fs + \Delta E_{int} \]
since $T = f$. Thus, since $\Delta E_{int} > 0$ since temperature (and thus internal energy) is increasing, the work done by friction must be \textit{less} than $fs$. 

\subsection{Center Mass Energy}
For some systems, the point of application does not move but the CM does. These are best analyzed with
\[ \int F \,dx_{cm} = \Delta K_{cm} \]
which becomes, for constant forces, 
\[ F_{ext}x_{cm} = \Delta K_{cm} \]

\subsection{First Law of Thermodynamics}
Our earlier energy relations did not account for heat (energy in transfer). This is corrected for with
\[ \Delta E_{tot} = Q + W \rightarrow \Delta K + \Delta U + \Delta E_{int} = Q + W\]
where $Q$ is the heat transferred into the system. 

\end{document}