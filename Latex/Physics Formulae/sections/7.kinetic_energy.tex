\documentclass[../PhysicsFormulae]{subfiles}
\begin{document}

\subsection{Kinetic Energy}
Kinetic energy is the energy of motion, defined as 
\[ K = \frac{1}{2}mv^2 \]

\subsection{Work}
Work involves external forces exerted through the point of application, given by
\[ W = \int \vec{F} \cdot \,d\vec{s} = F_x \Delta x + F_y \Delta y + F_z \Delta z \]
Work is related to kinetic energy through the Work-Energy Theorem, given by
\[ W = \Delta K \]
which holds if an object only has kinetic energy.

\subsubsection{Work by Spring Forces}
For a Hooke's Law force $F=-kx$, the work done by the spring is given by
\[ W = \int_{x_i}^{x_f} -kx \,dx = -\frac{1}{2}k(x_f^2 - x_i^2) \]

\subsubsection{Two Dimensional Work}
For a particle moving along a 2D curve, work is given by
\[ W = \int_i^f \vec{F} \cdot \,d\vec{s} = \int_i^f F\cos{\phi} \,ds = \int_i^f (F_x \,dx + F_y \,dy) \]
where $\vec{F} = F_x \hat{i} + F_y \hat{j}$, $d\vec{s} = dx \hat{i} + dy \hat{j}$, and $\phi$ is the angle between the vectors.

\subsection{Power}
Power is the rate at which work is done, instantaneously given by
\[ P = \frac{dW}{dt} = \frac{\vec{F} \cdot d\vec{s}}{dt} = \vec{F} \cdot \vec{v} \]
and thus average power is 
\[ P_{avg} = \frac{W}{t} \]

\subsection{Rotational Kinetic Energy}
The energy of rotational motion is given by
\[ K = \frac{1}{2}I\omega^2 \]

\subsection{Rotational Work}
Rotational work is given by
\[ W = \int \vec{F} \cdot \,d\vec{s} = \int (F\sin{\phi}) (r\,d\phi) = \int \uptau \,d\phi \]

\subsection{Rotational Power}
Rotational power is given by
\[ P = \frac{dW}{dt} = \uptau_z \frac{d\phi}{dt} = \uptau_z \omega_z \]

\subsection{Energy in Collisions}
In a perfectly inelastic collision with one object starting at rest, the ratio of final kinetic energy to initial kinetic energy is given by
\[ \frac{K_f}{K_i} = \frac{m_1}{m_1 + m_2} \]
which means the ratio of lost energy to initial energy is
\[ \frac{K_{lost}}{K_i} = \frac{m_2}{m_1 + m_2} \]
where $m_1$ is the mass of the initially moving particle and $m_2$ is the mass of the particle initially at rest.

\end{document}