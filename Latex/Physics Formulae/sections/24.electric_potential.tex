\documentclass[../PhysicsFormulae.tex]{subfiles}
\begin{document}

\subsection{Potential Energy}
Because the electrostatic force is conservative, it has a potential energy function given by 
\[ \Delta U = -\int_a^b \frac{1}{4\pi\epsilon_0} \frac{q_1q_2}{r^2} \; dr = \frac{1}{4\pi \epsilon_0}q_1q_2 \left(\frac{1}{b} - \frac{1}{a} \right) \]
Letting $a = \infty$ and $b = r$ and, by convention, allowing $U=0$ at $r = \infty$, we have 
\[ U = \frac{1}{4\pi\epsilon_0}\frac{q_1q_2}{r} \]

\subsection{Electric Potential}
The potential energy is proportional to the test charge; to hell with that, so we define \textit{electric potential} as
\[ \Delta V = \frac{\Delta U}{q_0} \]
Voltage is used interchangeably with electric potential, and it is measured in \textit{volts}, or joules per coulomb. Rearranging, we can also write 
\[ \Delta U = q \Delta V \]

Electric potential and electric field can be also connected via
\[ \Delta V = -\frac{W_{ab}}{q_0} = -\int_a^b \vec{E} \cdot d\vec{s} \]

Suppose we have a charge $q$ and a test charge $q_0$. We want to know how much work per unit charge is needed to move $q_0$ from $a$ to $b$ on the line joining the charges. This is
\[ V_b - V_a = \frac{q}{4\pi \epsilon_0} \left(\frac{1}{b} - \frac{1}{a} \right) \]

\subsection{Potential Due to an Electric Dipole}
The work per unit charge necessary to bring a test charge from infinity to a distance $r$ from the center of a dipole is given by 
\[ V = \frac{1}{4\pi \epsilon_0} \left(\frac{q}{r_+} + \frac{-q}{r_-} \right) \approx \frac{1}{4\pi \epsilon_0} \frac{p\cos{\theta}}{r^2} \]
when $r>>d$, using the approximation $r_- - r_+ \approx d\cos{\theta}$ and $r_-r_+ \approx r^2$. 

\subsection{Electric Potential of Continuous Charge Distributions}
For a uniform line of charge, the potential is approximately
\[ V = \frac{1}{4\pi \epsilon_0} \frac{q}{y} \]
for points along its perpendicular bisector on the y-axis. \bigskip

For a uniform ring of charge, 
\[ V = \frac{1}{4\pi\epsilon_0} \frac{2\pi \lambda R}{\sqrt{R^2 + z^2} } \]
for points along its axis of symmetry. \bigskip 

For a uniform disk of charge, by integrating the potential of rings of infinitesimal thickness, 
\[ V = \frac{\sigma}{2\epsilon_0} \left(\sqrt{R^2 + z^2} - |z| \right) \]

\subsection{Calculating Field from Potential}
Consider if you nudge a particle of charge $q$ in some direction. The external work you do is
\[ W = q\Delta V = -qE \Delta s \rightarrow E = -\frac{\Delta V}{\Delta s} \]
because the force you apply is opposite the force from the electric field. More generally, 
\[ \vec{E} = -\vec{\nabla} V = -\frac{\partial V}{\partial x} \hat{i} - \frac{\partial V}{\partial y} \hat{j} - \frac{\partial V}{\partial z} \hat{k} \]

\subsection{Equipotential Surfaces}
On an equipotential surface, where $V$ is constant, the electric field must do 0 work as a charge travels along the surface. Any given charge distribution may have an entire \textit{family} of charge distributions. \bigskip

By this logic, the electric field must be entirely perpendicular to the equipotential surface; otherwise, work would be done along it. 

\subsection{Potential of a Charged Conductor}
Consider the surface of a charged conductor. It is an equipotential because, by symmetry, all electric field lines emanate in a (perpendicular) radial direction to it. However, because the electric field is 0 inside the conductor, the potential must not change on the interior as well. Thus, we can conclude that the \textit{entire} conductor is an equipotential surface with 
\[ V = \frac{1}{4\pi \epsilon_0} \frac{q}{R} \]

\subsection{Corona Discharge}
Electric breakdown occurs when the electric field at some location exceeds the \textit{dielectric strength} of the fluid (air) around it. A specific form of breakdown is corona discharge, caused by sharp, pointy areas due to their high charge density. \bigskip

Qualitatively, consider two (distant) conducting spheres connected by a wire. They must be at equipotential, so
\[ \frac{V_2}{V_1} = \frac{\sigma_2 R_2}{\sigma_1 R_1} = 1 \rightarrow \sigma \propto \frac{1}{R} \]
At regions where the radius of curvature is very small (i.e., very sharp regions), the charge density is super duper high. And because $E \propto \sigma$ the electric field is very large as well, possibly even exceeding the dielectric strength of air and leading to corona discharge. 

\subsection{Electrostatic Accelerator}
Consider two concentric conducting spheres. If charge can be placed on the inner sphere consistently (perhaps through a small hole in the outer sphere), then a wire can repeatedly be connected between the spheres so that the charge flows to the outer surface, thereby increasing potential without limit (theoretically). Robert Van de Graaff utilized this principal to invent the electrostatic accelerator, known as a Van de Graaff today, with a moving belt that sprayed charge onto a conducting sphere. 

\end{document}