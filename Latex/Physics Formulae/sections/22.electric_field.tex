\documentclass[../PhysicsFormulae.tex]{subfiles}
\begin{document}

\subsection{Electric Field}
Electric field is defined as 
\[ \vec{E} = \lim_{q_0 \to 0}\frac{\vec{F}}{q_0} \]
where $q_0$ is a test charge at some location in space. Thus, for a point charge, 
\[ E = \frac{1}{4\pi \epsilon_0} \frac{|q|}{r^2} \]

\subsection{Electric Dipole}
An electric dipole is a configuration of two opposite charges, $+q$ and $-q$, separated by distance $d$. The electric dipole moment is then
\[ \vec{p} = q\vec{d} \]
where the vector $d$ is drawn from the negative to positive charge. (Note that this is opposite of the convention used in chemistry.) Generally, the field of a dipole is 
\[ \vec{E} = \vec{E}_+ + \vec{E}_- \]
On a perpendicular bisector of the line joining the charges, the field varies as
\[ E = \frac{1}{4\pi \epsilon_0} \frac{p}{x^3} \]
in the direction of $-\hat{d}$ for $x>>d$.
On the axis of the dipole, 
\[ E = \frac{1}{4\pi \epsilon_0} \frac{p}{z^3} \]
in the direction of $\hat{k}$ for $z>>d$. \bigskip 

The quadrupole, octupole, and other higher-order moments also exist. For a quadrupole, $E \propto r^{-4}$ and for an octupole, $E \propto r^{-6}$.

\subsection{Electric Field Lines}
Field lines emanate from positive charges and end on negative charges. The number of electric field lines per unit cross-sectional area is proportional to the strength of the field at that point. 

\subsection{Measuring Elementary Charge}
The earliest measurements of the quantized elementary charge were conducted by Robert Millikan as follows: oil drops of charge $q$ (assumed negative) were allowed to fall in a chamber, eventually reaching terminal speed $v$. Then, a downwards field $\vec{E}$ would be applied, sending electrons upwards until a new terminal speed $v'$ was reached. Thus, 
\[ m\vec{g} - b\vec{v} = 0 \]
\[ q\vec{E} + m\vec{g} - b\vec{v'} = 0 \]
Measurements of $v$ and $v'$ would then allow $q$ to be determined. Millikan discovered that all charges $q$ were multiples of an elementary charge $e$; that is, 
\[ q = ne \; \; n=0, \pm1, \pm2, \pm3 \cdots \]

\subsection{A Dipole in an Electric Field}
Suppose a dipole is placed in an external electric field, with $\vec{d}$ at an angle $\theta$ to $\vec{E}$. The torque on the dipole would be
\[ \tau = 2F\left(\frac{d}{2}\sin{\theta}\right) = pd\sin{\theta} \]
in vector form, 
\[ \vec{\tau} = \vec{p} \times \vec{E} \]
The work done by the external field is 
\[ W = \int_{\theta_0}^{\theta}\vec{\tau} \cdot d\vec{\theta} = pE(\cos{\theta} - \cos{\theta_0} ) \]
because the torque always tends to restore the dipole into alignment with the field, so the dot product is negative. \bigskip

The potential energy, allowing the reference angle to be $\theta_0 = 90^{\circ}$, is
\[ U = -\vec{p} \cdot \vec{E} \]

\subsection{Nuclear Model of the Atom}
One way to test the structure of an atom is by shooting a beam of positively charged particles towards it. By letting each projectile be significantly less massive than the atom but more massive than an electron, we ensure that only the electric force from the nucleus has a significant effect on the deflection of the beam. \bigskip

Suppose the horizontally-launched projectiles pass very close to the surface of the nucleus. Then, the maximum transverse velocity deflection is
\[ \Delta v = \frac{F}{m} \Delta t = \frac{1}{m} \left(\frac{Qq}{4\pi\epsilon_0 R^2} \right) \left( \frac{2R}{v} \right) \]
The maximum deflection angle would thus be 
\[ \theta = \frac{\Delta v}{v} \]
Ernest Rutherford first conducted this experiment with a stream of alpha particles. While many particles were deflected only marginally, some were deflected by angles greater than $90^{\circ}$, something impossible under the Thompson's Plum-Pudding Model (where electrons and protons were uniformly distributed throughout a sphere). He used this to develop the concept of a nucleus. 
\end{document}