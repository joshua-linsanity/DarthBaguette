\documentclass[../PhysicsFormulae.tex]{subfiles}
\begin{document}

There are three types of forces:
\begin{enumerate}
\item \underline{Compression}: pushing force is perpendicular to surface
\item \underline{Tension}: pulling force is perpendicular to surface
\item \underline{Shearing}: force is parallel to surface
\end{enumerate}
Solids support all three; liquids can support compression and (to some extent) tension, gases support none.

\subsection{Pressure}
Pressure is the perpendicular force per unit area, given by
\[ P = \frac{F_{\perp}}{A} \]
and the density is given by
\[ \rho = \frac{m}{V} \] 

\subsection{Bulk Modulus}
When we increase pressure, volume must decrease, and this tendency is measured by
\[ B = -\frac{\Delta P}{\Delta V / V} \]
where the units are $[B] = [P]$. The modulus of water is $B_w = 2.2 \times 10^9 \; Pa$, meaning it is not very compressible. 

\subsection{Pressure Variation}
The relation of pressure to height in a fluid is given by
\[ \frac{dP}{dy} = - \rho g \]
and for incompressible fluids this becomes
\[ P_2 - P_1 = -\rho g h \]
Using this, the force due to water on the walls of a pool is given by
\[ F = P_{av}A = \frac{\rho gh}{2} A = \frac{\rho g l h^2 }{2} \]
for a wall of dimensions $l \times h $. Atmospheric pressure is irrelevant; it equally affects both sides of the wall. \\
And also, we can solve for the density of any fluid if we know another's; that is, 
\[ \rho_1gy_1 = \rho_2gy_2 \rightarrow \rho_2 = \frac{2a}{2a+d}\rho_1 \]
where $a$ is the displacement of original fluid from equilibrium, $d$ is the current height differential, and we poured the fluid of $\rho_2$ into the fluid of $\rho_1$. \bigskip

For \textbf{atmospheric pressure}: from the ideal gas law $P \propto V^{-1}$ and density yields $\rho \propto V^{-1}$ and thus $P \propto \rho$. Thus, solving the DE involving $dP/dy$ yields
\[ P = P_0 e^{-\frac{\rho g}{P_0} h} = P_0 e^{-h/a} \]
where $a = P_0/\rho g$ and $h$ is altitude above Earth's surface. 

\subsubsection{Raindrop Pressure}
Given drop density $n$, radius $r$, and contact speed $v$, the pressure exerted on a horizontal surface is given by
\[ P = \frac{4}{3}\pi \rho nv^2r^3 \]

\subsection{Pascal's Principle}
Pascal's Principle says that for a static fluid, pressure applied is transmitted undiminished equally across the fluid. It holds for both incompressible and (after reaching equilibrium) compressible fluids. 

\subsection{Archimedes' Principle}
For a body in fluid, there is a buoyant (static lift) force coming from pressure differentials, given by
\[ B = \rho_{fl} V_{d} g = \frac{\rho_{fl}}{\rho_{obj}} Mg \]
the latter of which occurring only in full submersion. \\
Generally, for floating objects, 
\[ \frac{V_d}{V} = \frac{\rho_{obj}}{\rho_{fl}} \]

\subsubsection{Mercury Barometer}
Assuming negligible vapor pressure, the mercury barometer lets us measure atmospheric pressure, given by
\[ P_{atm} = \rho_m gh \]
and it turns out a column of mercury 760 mm tall is needed, thus defining the Torr. 

\subsubsection{Open Tube Manometer}
The open tube manometer is used to measure gauge pressure, or overpressure, given by
\[ P - P_0 = \rho g h \]

\subsection{Surface Tension}
Surface tension is defined as 
\[ \gamma = \frac{F}{L} \]
where $L$ is the distance over which the force acts. Manipulating this fraction gives us
\[ \gamma = \frac{Fh}{Lh} = \frac{U}{A} \]
where $h$ is the height. 

\subsubsection{Soap Bubble}
The energy stored by a soap bubble is given by
\[ U = \gamma (2A) = 8\gamma \pi r^2 \]
because there are two \textit{free surfaces} (inside and outside). \\
How much work is done in expanding the soap bubble by $\Delta A$?
\[ W = \gamma \Delta A \]
What is the pressure differential between inside and outside necessary to support the bubble?
\[ \Delta P = \frac{4\pi r \gamma}{\pi r^2} = \frac{4\gamma}{r} \]
which comes from considering the bubble as two hemispheres. 

\end{document}