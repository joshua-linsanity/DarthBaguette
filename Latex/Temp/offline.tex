\documentclass{article}
\usepackage[utf8]{inputenc}
\usepackage[margin=1in]{geometry}
\usepackage{amsmath}
\usepackage{amssymb}
\usepackage{fancyhdr}
\usepackage{graphicx}
\usepackage{subcaption}

\graphicspath{ {./images} }
\pagestyle{fancy}
\fancypagestyle{firstpage}{%
    \lhead{}
    \rhead{{\huge18}}
}
\setlength{\headheight}{17pt}% ...at least 16.96927pt


\begin{document}
    \thispagestyle{firstpage}
    \newpage
    \vspace*{\fill}
    \begin{center}
        \begin{minipage}{.6\textwidth}
            \begin{centering}
		{\huge Offline Problems - Mika\bigskip

                Physics 4A\bigskip

                Example by Joshua\bigskip

                Due November}

            \end{centering}
        \end{minipage}
    \end{center}
    \vfill

    \newpage
	\section{Hanging Brick Question}
	\textbf{Question.} Two identical uniform bricks of length $L$ are stacked on top of each other at the edge of a table. Find the maximum overhang length $L_{max}$ that the system can have beyond the edge of the table without toppling. Express your answer in terms of the brick length $L$, and make sure to clearly justify your answer! \bigskip

	\noindent\textbf{Solution.} Two stipulations:
	\begin{enumerate}
		\item Center of Mass needs to be at the very edge for maximum overhang
		\item Same force acts on each block, so their positions get the same weight
	\end{enumerate}

	Mathematically, this means $x_{cm} \geq L_{max}$. So, this becomes a matter of solving for $x_{cm}$:


	\[ x_{cm} = \frac{mL/2 + m(x + L/2)}{m + m} = \frac{L + x}{2} \]

	At the very least, we understand that $x \leq L/2$ because the top block must not exceed half its length overhanging for it to remain on the structure. \bigskip

	This implies that at its largest, $x_{cm} = \frac{L + L/2}{2} = \frac{3L}{4}$. So, applying our first condition,

	\[ \boxed{L_{max} = \frac{3L}{4}} \]


\end{document}


